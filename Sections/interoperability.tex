\section{PID interoperability with NDN}
%Short intro about it
%-Show our model

In this section, we propose a model shown in figure \ref{fig:sdc_model} to be demonstrated in our POC. The proposed model achieves PID interoperability within the NDN namespace and makes adding future PID types to be achieved easily.
\newline
\newline
Our model adheres to the following principles.

\begin{itemize}
    \item{Translation is transparent to the user}
    \item{Support for multiple PID types}
    \item{Extensible with future PID types with different naming schemes}
\end{itemize}

\begin{figure}[H]
\centering
\includegraphics[scale=0.25]{Images/PIDNDN_edit_final_5.png}
\caption{Proposed model for SeaDataCloud.}
\label{fig:sdc_model}
\end{figure}

\subsection{Interoperability}
%Way of doing it described in detail, describe the 'how' answer

%Our solution tackles the issue of Mousa and Karakannas where the client has to be updated.
TO-DO:
\newline
Explain model in detail.
PID consortium for pattern matching. PID consortium defines standards.

Pattern matching is used for matching the PID types. And they are here: ptr consortium.
ePIC Data Type Registry (testing)

Client types in PID and gets redirected automatically to either the PID server or NDN network.

We don not match “handle” or “handle:”, “doi” or “doi:“, “urn” or “urn:”. As that really depends on everey PID provider (KB does not do this, urn is mentioned in the URL but this can already be matched according to pid-consortium!) 

Take in account long NDN names as described by Yuan et al. \cite{yuan2012scalable} which is mentioned in section \ref{introduction-related-work}.

GW XML/JSON/or whatever parser to parse metadata that the PID resolver utilizes if needed for usage to fill in gaps of an NDN name, sometimes this is needed for missing name components that are needed to hierarchically divide the PID into NDN as researched by Olschanowsky et. al. in section related work. 
This depends on the PID provider how metadata is served.

Our proposed gateway is similar to The Names to Things resolver (N2T), which can also resolve different PID types, by stating and adding the PID type and the regex pattern. https://n2t.net/

!!! Using web resolver URLs in the GW code, that is why I gave examples of web resolvers for better understanding this and to give better insight in the PID syntax. !!!

Metadata, rules, web resolver link or not etc. all depending on PID provider and cloud itself.

Mention using ndnputchunks and ndncatchunks of ndn-tools for POC. Data is cached in memory and takes 3x the size for encoding (I have a source for this). Caching on intermediate routers seems only to take 1x the size as seen yesterday in my mini experiment. For a persistent file cache, compile file-server with NFD, used nfds or ...