%\section{System prototype}
\section{PID interoperability with NDN}
\label{pid-poc}

% introduction of what's happening here

%Short intro about it
%-Show our model
%\subsection{PID interoperability}
In this section, we will propose a design, which achieves PID interoperability with the NDN namespace and makes it feasible to add future PID types to answer our research question in section \ref{introduction-research-question}. Our design is based on related work done by Karakannas \cite{icn-bd}, by avoiding the PID to NDN translation on client side \cite{icn-bd}. The pattern matching method was based on related work by Mousa for identifying different PID type schemas \cite{ndn-app-aware}. The research done by Olschanowsky et. al. was used for deriving NDN names from metadata \cite{ndn-man}.
We have reimplemented the principles of NaaS4PID in order to demonstrate the PID interoperability extensibility feature.
We combined the aforementioned related work and extended it with an extendable interoperability feature, we attempt to make the translation transparent to the user and support multiple PID types. The proposed design for our solution is illustrated in figure \ref{fig:sdc_model}.
Our design adheres to the following aforementioned principles, which will be discussed in more detail in this section.
 
\begin{itemize}
    \item{Translation is transparent to the user.}
    \item{Support for multiple PID types.}
    \item{Extensible with future PID types with different naming schemas.}
\end{itemize}

\begin{figure}[H]
\centering
\includegraphics[width=\textwidth]{Images/PIDtoNDN11.png}
\caption{NDN virtual function based 
planning for achieving PID interoperability}
\label{fig:sdc_model}
\end{figure}

Our proposed design consists of the following components, each with their own functionality; the PID server, the PID to NDN gateway and the client. The general idea is that a user enters a PID of the object that the user want to retrieve at the client and gets back the requested object as shown in figure \ref{fig:sdc_model}. The retrieval of an object depends if the object is already published in the NDN or not, which is further described in sections \ref{client} and \ref{gw}.
%If it is published in NDN, then the client retrieves the object from NDN. If it is not published in NDN, the object is retrieved from the PID server. This is further described in section \ref{gw}.  

% integrate with section 3, planning concept
\subsection{Proof of concept} 
This section discusses the implementation of our design in a proof of concept.
In our proof of concept, the components mentioned in section \ref{pid-poc} are conceptually part of the NDN we have setup in our high-level network design, which will be discussed in section \ref{planning-architecture}.

\subsubsection{PID server}
The PID server is maintained by the PID provider, which identifies objects of a particular PID type. The PID types we cover in our proof of concept are highlighted in section \ref{pid-types}. For our proof of concept we have set up a Handle PID server with Cordra software \cite{cor} to store and identify our data. We got allotted the Handle prefix \texttt{20.5000.481} by the Handle registry. We use this prefix for object identification. The Handles stored on our Handle PID server are resolved by the Handle System (\texttt{http://hdl.handle.net}. In addition to this, we also used the resolver of the national library of the Netherlands for resolving URN's as well as PANGAEA, which resolves DOI's. 

\subsubsection{Client}\label{client}
The role of the client is to transparently make the user either retrieve the requested object from the corresponding PID provider or from the NDN. This will be further described in section \ref{gw}. The client receives the PID based on the user's input, which can be any kind of PID type. The client then opens a socket and sends the user's request to the gateway, which corresponds with step 1 and 2 in figure \ref{fig:sdc_model}. After the gateway does the translation, it send back either a translated NDN name from the PID or a link to the PID server to the client. This depends whether the object is published in the NDN or not. If it is, the client receives the NDN name. If not, it receives the link to the PID server for retrieving the object. This corresponds to step 6a in figure \ref{fig:sdc_model}.

If an NDN name is sent back to the client, the client will request the object from the NDN\footnote{\url{https://github.com/AquaL1te/rp2/blob/master/Scripts/ndn_client.py}}. This is shown as step 7a and step 8 till 10 in figure \ref{fig:sdc_model}. For object retrieval in NDN we used the tool \texttt{ndncatchunks} part of the \texttt{ndn-tools} software, which makes use of the libraries of the \gls{cxx} application \cite{ndn-tools}. 
If the object is not in NDN and a PID link is sent back, the client requests the object by its PID link by sending the request to the PID server\footnote{\url{https://github.com/AquaL1te/rp2/blob/master/Scripts/pid_client.py}}. This corresponds to step 7c in figure \ref{fig:sdc_model}. Object retrieval is done at client side, otherwise the gateway has to retrieve the object first before sending it to the client. 
%As in the case of retrieving the object from the PID server to publish it in NDN, 
The client would have to wait for the gateway till it has retrieved and cached the object in NDN. 
In the case that the object is already published in NDN, the gateway server only concerns itself with translation and sending back the NDN name to the client. This eliminates unnecessary load on the gateway.

Retrieving an object either from the PID server or from NDN should happen transparently for the user. It is possible to accomplish this by combining the code for retrieving the object from the PID server with the code we have used to retrieve the object from NDN at the client side. Furthermore, a conditional statement needs to be added at the gateway to check if the object is already published in NDN. This is further described in section \ref{gw}.
%can be done with the \texttt{ndnping} tool \cite{ndn-tools} for example as described in section \ref{gw}.

As a result, if transparency is implemented the user might not even be aware of an NDN. The user only specifies a PID as input for the client, without specifying the link of the web resolver that resolves the PID. The user gets redirected automatically by the client to either the PID server or NDN. No further user input is needed as everything is taken care of by the gateway, which will also be discussed in section \ref{gw}.
%Unlike previous designs, which require user input after the client has translated the PID to NDN name \cite{ndn-app-aware}. 

\subsubsection{Gateway}\label{gw}
The gateway used in our design follows the principles of Karakannas by avoid doing the PID to NDN translation on client side. In the case of doing the translation on client side, the client software needs to be updated every time a new PID type is introduced. For our proof of concept, we designed a translation server called the PID to NDN gateway. The PID to NDN gateway implements the translation of different PID types and sends the translated name back to the client. Furthermore, we identify PID types based on pattern matching as described by Mousa \cite{ndn-app-aware}. We have also looked at how the \gls{n2t} resolver deals with different PID types. The \gls{n2t} resolves different PID types by stating the PID type that needs to be implemented along with the pattern of the PIDs' schema \cite{n2t}.

%Translation
The gateway is responsible for translating a PID to NDN name and checks if the requested object is already published in NDN. 
The first responsibility is translating the PID it receives from the client to an NDN name, which can be of any PID type. In our proof of concept we have implemented the Handle PID type schema of the Handle PID server we setup. In addition to this, we have also implemented the URN PID type schema of the national library of the Netherlands, as well as the DOI type schema of PANGAEA. The gateway receives a PID from the client, without the link of the corresponding web resolver of the PID type. Based on pattern matching of the PID type schema\footnote{\url{https://github.com/AquaL1te/rp2/blob/master/Scripts/pid_server.py\#L58-L62}}, the gateway detects what kind of PID type it has to deal with. Then, the associated function is called to translate the PID to NDN name\footnote{\url{https://github.com/AquaL1te/rp2/blob/master/Scripts/pid_server.py\#L17-L37}} and appends the corresponding link of the web resolver of the PID type it receives. 
%Pattern matching is done based on the patterns of 
The patterns of most standardized PID type schemas are maintained in the ePIC \gls{dtr} \cite{dtr} and can be used for implementation in the gateway we propose. 

%Naming:
Furthermore, the second responsibility of the gateway is to check if the object is already published in NDN.
%This can be done with a ping to the object with \texttt{ndnping}, part of the \texttt{ndn-tools} software \cite{ndn-tools} used in our proof of concept. To use this, a \texttt{ndnpingserver} has to be setup for the objects with the NDN testing software we have utilized. The presence of the object can also be detected with \texttt{ndncatchunks} by catching the error exception, which is thrown if the object is not published by \texttt{ndnputchunks}.
If the object is available in NDN, the gateway sends the translated NDN name back to the client. The client then retrieves the object from NDN. This is shown in figure \ref{fig:seq_ndn}, where the Handle PID type is used as an example. If the object is not available in NDN, the gateway sends back the PID link to the client and caches the object in NDN. 
%The client then retrieves the object by its PID link from the PID server as shown in figure \ref{fig:seq_pid}. 

The PID link that is sent back to the client contains the PID and the link to the corresponding PID web resolver. This happens before the gateway retrieves the object to cache it in NDN, otherwise the client needs to wait for this as discussed in section \ref{client}. For our proof of concept we use the tool \texttt{ndnputchunks}, which is part of the \texttt{ndn-tools} software for caching objects in the NDN network \cite{ndn-tools}. Transparency can be achieved, if the code we used in our setup to retrieve the object from NDN is combined with the code we used to retrieve the object from the PID server. This is done by including a conditional statement to check if the object is already published in NDN with \texttt{ndnping} as we mentioned. 
To translate the matched PID type to an NDN name, one has to take into account the schema of the PID type and the hierarchical way it has to be divided in. In our proof of concept a prefix is added before the PID name, for deriving an NDN name from each discussed PID type. Such as \texttt{/ndn/handle} for Handle objects, \texttt{/ndn/doi} for DOI objects and \texttt{/ndn/urn} for URN objects. The delimiters of the PID types are replaced by a slash \texttt{("/")}. In PANGAEA, specific columns and parameters can be requested to retrieve a particular part from an object\footnote{\url{https://doi.pangaea.de/10.1594/PANGAEA.842227&columns=1,,2,3&filterParameterValue=Station,TARA_100}}. The columns and parameters can also be translated to an NDN name\footnote{\url{/ndn/doi/10.1594/PANGAEA.842227/attrib+ndn+1,2,3+Station,TARA_100}} \cite{ndn-app-aware}. This shows that not only the delimiters are replaced to translate it into an NDN name. This is done by hierarchically dividing it in \texttt{/attrib+ndn} and appending the columns and parameters with a plus ("+") (where the requested columns and parameters are separated by a comma(",")). Furthermore, the web resolver link is also stripped after translation. 

The PID link that is sent back to the client contains the PID and the link to the corresponding PID web resolver. This happens before the gateway retrieves the object to cache it in NDN, otherwise the client needs to wait for this as discussed in section \ref{client}. For our proof of concept we use the tool \texttt{ndnputchunks}, which is part of the \texttt{ndn-tools} software for caching objects in the NDN \cite{ndn-tools}. Transparency can be achieved, if the code we used in our setup to retrieve the object from NDN is combined with the code we used to retrieve the object from the PID server. This is done by including a conditional statement to check if the object is already published in NDN. This can be done with \texttt{ndnping} for example, which is also part of the \texttt{ndn-tools} testing software that we have utilized. To use \texttt{ndnping}, a \texttt{ndnpingserver} has to be setup for the objects. Another option is to catch the error, which is thrown by \texttt{ndncatchunks} if the object is not published by \texttt{ndnputchunks}. 
To translate the matched PID type to an NDN name, one has to take into account the schema of the PID type and the hierarchical way it has to be divided in. In our proof of concept a prefix is added before the PID name, for deriving an NDN name from each discussed PID type. Such as \texttt{/ndn/handle} for Handle objects, \texttt{/ndn/doi} for DOI objects and \texttt{/ndn/urn} for URN objects. This complies with the related work we have discussed in section \ref{introduction-related-work}. The delimiters of the PID types are replaced by a slash \texttt{("/")}. In PANGAEA, specific columns and parameters can be requested to retrieve a particular part from an object\footnote{\url{https://doi.pangaea.de/10.1594/PANGAEA.842227&columns=1,,2,3&filterParameterValue=Station,TARA_100}}. The columns and parameters can also be translated to an NDN name\footnote{\url{/ndn/doi/10.1594/PANGAEA.842227/attrib+ndn+1,2,3+Station,TARA_100}} \cite{ndn-app-aware}. This shows that not only the delimiters are replaced to translate it into an NDN name. This is done by hierarchically dividing it in \texttt{/attrib+ndn} and appending the columns and parameters with a plus \texttt{("+")} (where the requested columns and parameters are separated by a comma \texttt{(","))}. 
Afterwards, the requested data chunk is stored in NDN with the parameter and column attributes. The next time someone wants to retrieve that part, it will be already available. Another solution would be to request the whole object and hierarchically divide the attributes in an NDN name by some kind of rules. This leads to having all chunks of a object available in NDN, so the next time a part of the object can be requested.
Furthermore, the web resolver link is also stripped after translation.


The web resolver link is not used for deriving the NDN name.
By excluding the web resolver link from the NDN name, duplications will not occur in NDN as the NDN name is only derived from the PID. A PID always translates to the same name in NDN this way. 
If the PID object is moved to another web resolver, only the link of the web resolver has to be updated in the gateway. 

In listing \ref{lst:hdl_ndn}, the translation of a Handle PID to an NDN name is shown, the Handle name is hierarchically divided into its PID type, authority and sub-authority.
\vspace{1em}
\begin{lstlisting}[frame=single,gobble=0,basicstyle=\scriptsize\ttfamily,caption={Handle PID to NDN name}\label{lst:hdl_ndn}]
<user>@consumer-1:~/python-ndn$ python3 server_pid.py
Waiting for client
PID from connected user: 20.500.481/sub-auth/object1
PID type: Handle
NDN name from Handle: /ndn/handle/20.500.481/sub-auth/object1
\end{lstlisting}

The translation of a URN PID to an NDN name is shown in listing \ref{lst:urn_ndn} for the ANP collection maintained by the national library of the Netherlands as discussed in section \ref{urn-1}. The national library of the Netherlands chose to assign a PID based on the year when the object has been published. 
This way, objects in NDN can be hierarchically divided in year, months or days for example.
\vspace{1em}
\begin{lstlisting}[frame=single,gobble=0,basicstyle=\scriptsize\ttfamily,caption={URN PID to NDN name}\label{lst:urn_ndn}]
<user>@consumer-1:~/python-ndn$ python3 server_pid.py
Waiting for client
PID from connected user: anp:1938:10:01:2:mpeg21
PID type: URN
NDN name from URN: /ndn/urn/anp/1938/10/01/2/mpeg21
\end{lstlisting}

Metadata can be used if missing gaps need to be filled in for dividing the PID in the NDN hierarchy, as described by Olschanowsky et. al. \cite{ndn-clim} in section \ref{introduction-related-work}. 
Filling in missing gaps highly depends on which way the metadata is served by the providers of the different PID types as discussed in section \ref{pid-types}. Thus, if metadata is used, a parser has to be implemented in the gateway. This could be a XML, JSON or an other kind of parser depending on the PID provider. In our proof of concept we have implemented a XML parser for URN's and a JSON parser for Handles.

There is no address exhaustion problem in NDN as the NDN namespace is unbounded \cite{ndn-nspace}. But worth mentioning is to keep in mind that using long NDN names degrades performance with many interests as described by Yuan et al. \cite{yuan2012scalable} in section \ref{introduction-related-work}.

\begin{figure}[H]
%\flushleft
    \centering
%    \makebox[0pt]
    \caption{Handle PID request\label{fig:seq_pid}}
\end{figure}

\begin{figure}[H]
\includegraphics[scale=0.75]{Images/ndn_req.png}
\caption{Handle NDN request}
\label{fig:seq_ndn}
\end{figure}


% merge with results
\subsection{Results}\label{results-pid}
The outcome of implementing our design in a proof of concept shows that our principles can be adhered. Making the translation and object retrieval transparent to the user is possible. Users should not have to concern themselves whether to retrieve the object through NDN or the PID server. This is due to gateway's responsibility for PID to NDN translation and the object retrieval, which is taken care of by the client. 
%This can be achieved by combining the code we used for retrieving the object from either the PID server or NDN. 
Translation is achieved by first recognizing the PID type based on pattern matching and then hierarchically divide the PID to an NDN name. Support for multiple PID types is also achieved by adding the schema of the PID types at the gateway, which makes it also easily extensible to support future PID types. By adding PID types at the gateway, we overcome the hurdle of updating the client software with the schema of newly introduced PID types each time when a new PID type is introduced.




%\subsection{Deploying an NDN network (McCabe and TOSCA)}
%\label{planning-deploying}
% refer to sections where architecture and design is defined
%Now that the network analysis, design and architecture are defined, a deployment strategy is needed. The high-level design (figure \ref{fig:high-level-network-design}) needs to become deployable with a scalable method. Scalable in this context means that a single deployment strategy can be used for different cloud providers. Furthermore, as defined in the scope (section \ref{introduction-scope}) if the infrastructure scales out, the effort for managing a larger infrastructure should be equal. As described in section \ref{overview-tosca}, TOSCA is a standard to describe the complete life cycle of an infrastructure. Having a single set of template descriptions for deployment benefits portability and reproducibility of an infrastructure on different cloud providers.

%\begin{figure}[H]
%\centering
%\includegraphics[width=\columnwidth]{Images/tosca-diagram.png}
%\caption{TOSCA diagram.}
%\label{fig:tosca-diagram}
%\end{figure}

%In figure \ref{fig:tosca-diagram}, a TOSCA diagram is illustrated. This diagram represents an abstract template description of the TOSCA relationships, in which the grey rectangular boxes are the core scalability factors. As described in section \ref{overview-tosca}, TOSCA consists out of several types; nodes, relationships and interfaces. The scaling properties are highlighted in the rectangular areas. The left area, highlighted as 'scaling in/out resources' contains a dependency chain of several virtual NDN functions. This dependency chain is also depicted numerically. Before a pod can be deployed on Kubernetes (step 2 to 5), a VM needs to exist (step 1). This is described by the 'dependsOn' relationship. Furthermore, with the requirements defined in section \ref{planning-requirements}, input constraints are described. These constraints are used by the orchestrator to make sure that the NDN infrastructure has sufficient resources available to operate. Once a VM is deployed, the dependency for Kubernetes is satisfied, thus Kubernetes can then be setup (step 2). Kubernetes can then deploy pods by the use of interfaces (step 3). These interfaces feed the containers with environment variables such as the gateway, a list of routes, the transport protocol for NDN, the NDN strategies and on which Kubernetes node this pod should run. The environment variables are given to the interface via the TOSCA inputs. These environment variables are then used by scripts that run inside the pods to setup NDN. Several constraints are set for these environment variables such as which valid transport protocols can be used for NDN, which NDN strategies are valid and which nodes are available. These constraints are defined with e.g. 'valid\_values' or 'greater\_than' definitions. These constraints help to guide the orchestrator to verify the inputs that are given for the template description. As illustrated in the second gray area 'scaling in/out the application', several pods can be instantiated (step 5a, 5b and 5c) from the image (step 4). These pods enable the virtual NDN functions as described in section \ref{planning-architecture}. These pods establish the NDN network and therefore are connected via the 'connectsTo' relationship. This network expands over to other Kubernetes nodes in the cluster by the use of the Kubernetes built-in overlay network.

%\subsection{PID interoperability architecture design}
%The discussed architecture for our proposed design in section \ref{planning-architecture} is shown in figure \ref{fig:sdc_model}. This section will explain more about the different components used in our proposed design.

%\begin{figure}[H]
%\centering
%\includegraphics[width=\textwidth]{Image%s/PIDtoNDN10.png}
%\caption{NDN virtual function based 
%planning for achieving PID interoperability}
%\label{fig:sdc_model}
%\end{figure}


%\subsubsection{PID server}
%The PID server is maintained by the PID provider, which identifies objects of a particular PID type. The PID types we cover in our proof of concept are highlighted in section \ref{pid-types}. For our proof of concept we have set up our own Handle PID server with Cordra software \cite{cor}. We got allotted the Handle prefix \texttt{20.5000.481} by the Handle registry. We use this prefix for object identification. In addition to this, we also used the resolver of the national library of the Netherlands for resolving URN's as well as PANGAEA, which resolves DOI's. 

%\subsubsection{Client}\label{client}
%The role of the client is to transparently make the user either retrieve the requested object from the corresponding PID provider or from the NDN network. This will be further described in this section \ref{gw}. The client receives the PID based on the user's input, which can be any kind of PID type. The client then opens a socket and sends the user's request to the gateway, which corresponds with step 1 and 2 in figure \ref{fig:sdc_model}. After the gateway does the translation, it send back either a translated NDN name from the PID or a link to the PID server to the client. This depends whether the object is published in the NDN network or not. If it is, the client receives the NDN name. If not, it receives the link to the PID server for retrieving the object. This corresponds to step 6a in figure \ref{fig:sdc_model}.

%If an NDN name is sent back to the client, the client will request the object from the NDN network\footnote{\url{https://github.com/AquaL1te/rp2/blob/master/Scripts/ndn_client.py}}. This is shown as step 7a and step 8 till 10 in figure \ref{fig:sdc_model}. For object retrieval in NDN we used the tool \texttt{ndncatchunks} part of the \texttt{ndn-tools} software \cite{ndn-tools}, which makes use of the libraries of the NDN-CXX application \cite{ndn-tools}. 
%If the object is not in NDN and a PID link is sent back, the client requests the object by its PID link by sending the request to the PID server\footnote{\url{https://github.com/AquaL1te/rp2/blob/master/Scripts/pid_client.py}}. This corresponds to step 7c in figure \ref{fig:sdc_model}. Object retrieval is done at client side, otherwise the gateway has to retrieve the object first before sending it to the client. 
%As in the case of retrieving the object from the PID server to publish it in NDN, 
%The client would have to wait for the gateway till it has retrieved and cached the object in NDN. 
%In the case that the object is already published in NDN, the gateway server only concerns itself with translation and sending back the NDN name to the client. This eliminates unnecessary load on the gateway.

%Retrieving an object either from the PID server or from NDN should happen transparently for the user. It is possible to accomplish this by combining the code for retrieving the object from the PID server with the code we have used to retrieve the object from NDN at the client side. Furthermore, a conditional statement needs to be added at the gateway to check if the object is already published in NDN. This is further described in section \ref{gw}.
%can be done with the \texttt{ndnping} tool \cite{ndn-tools} for example as described in section \ref{gw}.

%As a result, if transparancy is implemented the user might not even be aware of an NDN network. The user only specifies a PID as input for the client, without specifying the link of the web resolver that resolves the PID. The user gets redirected automatically by the client to either the PID server or NDN network. No further user input is needed as everything is taken care of by the gateway, which will be discussed below in section \ref{gw}.
%Unlike previous designs, which require user input after the client has translated the PID to NDN name \cite{ndn-app-aware}. 

%\subsubsection{Gateway}\label{gw}
%The gateway used in our design follows the principles of Karakannas by avoid doing the PID to NDN translation on client side. In the case of doing the translation on client side, the client software needs to be updated every time a new PID type is introduced. For our proof of concept, we designed a translation server called the PID to NDN gateway. The PID to NDN gateway implements the translation of different PID types and sends the translated name back to the client. Furthermore, we identify PID types based on pattern matching as described by Mousa \cite{ndn-app-aware}. We have also explored the N2T resolver, which resolves different PID types by stating the PID type that needs to be implemented along with the pattern of the PIDs' schema \cite{n2t}.

%Translation
%The gateway is responsible for translating a PID to NDN name and checks if the requested object is already published in NDN. 
%The first responsibility is translating the PID it receives from the client to an NDN name, which can be of any PID type. In our proof of concept we have implemented the Handle PID type schema of the Handle PID server we setup. In addition to this, we have also implemented the URN PID type schema of the national library of the Netherlands, as well as the DOI type schema of PANGAEA. The gateway receives a PID from the client, without the link of the corresponding web resolver of the PID type. Based on pattern matching of the PID type schema\footnote{\url{https://github.com/AquaL1te/rp2/blob/master/Scripts/pid_server.py\#L58-L62}}, the gateway detects what kind of PID type it has to deal with. Then, the associated function is called to translate the PID to NDN name\footnote{\url{https://github.com/AquaL1te/rp2/blob/master/Scripts/pid_server.py\#L17-L37}} and appends the corresponding link of the web resolver of the PID type it receives. 
%Pattern matching is done based on the patterns of 
%The patterns of most standardized PID type schemas are maintained in the ePIC DTR \cite{dtr} and can be used for implementation in the gateway we propose. 

%Naming:
%Furthermore, the second responsibility of the gateway is to check if the object is already published in NDN.
%This can be done with a ping to the object with \texttt{ndnping}, part of the \texttt{ndn-tools} software \cite{ndn-tools} used in our proof of concept. To use this, a \texttt{ndnpingserver} has to be setup for the objects with the NDN testing software we have utilized. The presence of the object can also be detected with \texttt{ndncatchunks} by catching the error exception, which is thrown if the object is not published by \texttt{ndnputchunks}.
%If the object is available in NDN, the gateway sends the translated NDN name back to the client. The client then retrieves the object from NDN. This is shown in figure \ref{fig:seq_ndn}, where the Handle PID type is used as an example. If the object is not available in NDN, the gateway sends back the PID link to the client and caches the object in NDN. 
%The client then retrieves the object by its PID link from the PID server as shown in figure \ref{fig:seq_pid}. 
%The PID link that is sent back to the client contains the PID and the link to the corresponding PID web resolver. This happens before the gateway retrieves the object to cache it in NDN, otherwise the client needs to wait for this as discussed in section \ref{client}. For our proof of concept we use the tool \texttt{ndnputchunks}, which is part of the \texttt{ndn-tools} software for caching objects in the NDN network \cite{ndn-tools}. Transparency can be achieved, if the code we used in our setup to retrieve the object from NDN is combined with the code we used to retrieve the object from the PID server. This is done by including a conditional statement to check if the object is already published in NDN. This can be done with \texttt{ndnping} for example, which is also part of the \texttt{ndn-tools} testing software that we have utilized. To use \texttt{ndnping}, a \texttt{ndnpingserver} has to be setup for the objects. Another option is to catch the error, which is thrown by \texttt{ndncatchunks} if the object is not published by \texttt{ndnputchunks}. 
%To translate the matched PID type to an NDN name, one has to take into account the schema of the PID type and the hierarchical way it has to be divided in. In our proof of concept a prefix is added before the PID name, for deriving an NDN name from each discussed PID type. Such as \texttt{/ndn/handle} for Handle objects, \texttt{/ndn/doi} for DOI objects and \texttt{/ndn/urn} for URN objects. This complies with the related work we have discussed. The delimiters of the PID types are replaced by a slash \texttt{("/")}. In PANGAEA, specific columns and parameters can be requested to retrieve a particular part from an object\footnote{\url{https://doi.pangaea.de/10.1594/PANGAEA.842227&columns=1,,2,3&filterParameterValue=Station,TARA_100}}. The columns and parameters can also be translated to an NDN name\footnote{\url{/ndn/doi/10.1594/PANGAEA.842227/attrib+ndn+1,2,3+Station,TARA_100}} \cite{ndn-app-aware}. This shows that not only the delimiters are replaced to translate it into an NDN name. This is done by hierarchically dividing it in \texttt{/attrib+ndn} and appending the columns and parameters with a plus \texttt{("+")} (where the requested columns and parameters are separated by a comma \texttt{(","))}. Furthermore, the web resolver link is also stripped after translation.

%The web resolver link is not used for deriving the NDN name.
%By excluding the web resolver link from the NDN name, duplications will not occur in NDN as the NDN name is only derived from the PID. A PID always translates to the same name in NDN this way. 
%If the PID object is moved to another web resolver, only the link of the web resolver has to be updated in the gateway. 

%In listing \ref{lst:hdl_ndn}, the translation of a Handle PID to an NDN name is shown, the Handle name is hierarchically divided into its PID type, authority and sub-authority.
%\vspace{1em}
%\begin{lstlisting}[frame=single,gobble=0,basicstyle=\scriptsize\ttfamily,caption={Handle PID to NDN name}\label{lst:hdl_ndn}]
%<user>@consumer-1:~/python-ndn$ python3 server_pid.py
%Waiting for client
%PID from connected user: 20.500.481/sub-auth/object1
%PID type: Handle
%NDN name from Handle: /ndn/handle/20.500.481/sub-auth/object1
%\end{lstlisting}

%The translation of a URN PID to an NDN name is shown in listing \ref{lst:urn_ndn}. The national library of the Netherlands chose to assign a PID based on the year when the object has been published. 
%This way, objects in NDN can be hierarchically divided in year, months or days for example.
%\vspace{1em}
%\begin{lstlisting}[frame=single,gobble=0,basicstyle=\scriptsize\ttfamily,caption={URN PID to NDN name}\label{lst:urn_ndn}]
%<user>@consumer-1:~/python-ndn$ python3 server_pid.py
%Waiting for client
%PID from connected user: anp:1938:10:01:2:mpeg21
%PID type: URN
%NDN name from URN: /ndn/urn/anp/1938/10/01/2/mpeg21
%\end{lstlisting}

%Metadata can be used if missing gaps need to be filled in for dividing the PID in the NDN hierarchy, as described by Olschanowsky et. al. \cite{ndn-clim} in section \ref{introduction-related-work}. 
%Filling in missing gaps highly depends on which way the metadata is served by the providers of the different PID types as discussed in section \ref{pid-types}. Thus, if metadata is used, a parser has to be implemented in the gateway. This could be a XML, JSON or an other kind of parser depending on the PID provider. In our proof of concept we have implemented a XML parser for URN's and a JSON parser for Handles.

%There is no address exhaustion problem in NDN as the NDN namespace is unbounded \cite{ndn-nspace}. But worth mentioning is to keep in mind that using long NDN names degrades performance with many interests as described by Yuan et al. \cite{yuan2012scalable} in section \ref{introduction-related-work}.

%\begin{figure}[H]
%\flushleft
%    \centering
%    \makebox[0pt]{%
%    \includegraphics[width=\textwidth]{Images/pid_seq5.png}
    %}
%    \caption{Handle PID request\label{fig:seq_pid}}
%\end{figure}

%\begin{figure}[H]
%\includegraphics[scale=0.75]{Images/ndn_req.png}
%\caption{Handle NDN request}
%\label{fig:seq_ndn}
%\end{figure}

% merge with pid setup
%\subsection{Proof of concept}
%\label{planning-poc}
%With the methodology defined, in which scalability and performance requirements are met and a method for deployment is described, a proof of concept was used to test the methodology. The orchestrators mentioned in section \ref{overview-tosca} are still in a prototype phase. Therefore, in our proof of concept we deployed the VMs and Kubernetes nodes manually. In practice the life cycle of the Kubernetes pods are managed by a TOSCA orchestrator. Without having a TOSCA orchestrator available, steps 2 through 5 in figure \ref{fig:tosca-diagram} were be carried out by Kubernetes exclusively. This was done by defining the configuration properties\footnote{\url{https://github.com/AquaL1te/rp2/blob/master/Kubernetes/expanded-cluster.yml}} of the pods manually. These properties include the NDN function name, e.g. router, producer or consumer. And also includes the routes (NDN prefixes) and the associated NDN face with the transport protocol to use (TCP or UDP). These parameters were then inserted into the NDN FIB by the scripts that were executed inside the pod\footnote{\url{https://github.com/AquaL1te/rp2/blob/master/Docker/producer/docker-entrypoint.sh}}. The NDN strategies were also configured by these scripts. Furthermore, if it is not defined where a pod should be running, Kubernetes will make this decision itself, based on the known resources in the Kubernetes cluster. If for example a Kubernetes node has more memory to spare than other nodes, then Kubernetes will likely decide to spawn the pod there. This Kubernetes node could potentially run in a cloud provider, located in another geographical area. Since the purpose is to provide data distribution through the use of NDN, locality becomes a key factor. Therefore, a pod is specifically assigned to a Kubernetes node in order to provide in-network caching in a specific geographical area.

% merge with pid results
%\subsection{Results}
%As a result, the NDN infrastructure life cycle can be managed from Kubernetes. Our proof of concept lacks a TOSCA orchestrator. Therefore, scaling in or out resources to other cloud provider is not demonstrated. However, scaling in or out the NDN application is demonstrated. This method allows to reconfigure an NDN infrastructure in NVF-style by interacting with Kubernetes as the orchestrator. Therefore, the YAML configuration of Kubernetes acts as the TOSCA template description and Kubernetes, which executes this configuration, acts as the orchestrator. In practice the Kubernetes configuration would be generated based on the TOSCA template descriptions and e.g. DRIP would act as the orchestrator. In effect, the NDN containers are spawned as NVFs which provides scalable management of the NDN network.

%The outcome of implementing our design %for PID interoperability 
%in a proof of concept shows that our principles can be adhered. Making the translation transparent to the user is possible, which means that the user may not be aware of receiving the object from the PID server or NDN. This is due to gateway's responsibility for PID to NDN translation and the object retrieval, which is taken care of by the client. This can be achieved by combining the code we used for retrieving the object from either the PID server or NDN. Translation is achieved by first recognizing the PID type based on pattern matching and then hierarchically divide the PID to an NDN name. Support for multiple PID types is also achieved by adding the schema of the PID types at the gateway, which makes it also easily extensible to support future PID types. By adding PID types at the gateway, we overcome the hurdle of updating the client software with the schema of newly introduced PID types each time when a new PID type is introduced.