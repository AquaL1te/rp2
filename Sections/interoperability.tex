\section{PID interoperability with NDN}\label{pid-poc}
%Short intro about it
%-Show our model

In this section, we describe an implementation for our proof of concept which follows the design of our model in figure \ref{fig:sdc_model}. 
As we did not have access to the NaaS4PID solution described by Zhao et. al. to extend the their solution, we have created this design. The proposed design achieves PID interoperability within the NDN namespace and makes adding future PID types to be achieved easily. Our design adheres to the following principles, which will be discussed in more detail in this section.

\begin{itemize}
    \item{Translation is transparent to the user}
    \item{Support for multiple PID types}
    \item{Extensible with future PID types with different naming schemes}
\end{itemize}

\begin{figure}[H]
\centering
\includegraphics[width=\textwidth]{Images/PIDtoNDN10.png}
\caption{NDN virtual function based 
planning for achieving PID interoperability}
\label{fig:sdc_model}
\end{figure}

\subsection{Proof of concept}
The design that we specified to be implemented in our proof of concept consists of the following components, each with their own functionality; the PID server, the PID to NDN gateway and the client. In our proof of concept, these components are conceptually part of the NDN network we have setup. The general idea is that a user enters a PID of the object that the user want to retrieve at the client and gets back the requested object as shown in figure \ref{fig:sdc_model}. The retrieval of an object depends if the object is already published in the NDN network or not. If it is published in NDN, then the client retrieves the object from NDN. If it is not published in NDN, the object is retrieved from the PID server. This is further described in section \ref{gw}.  

\subsubsection{PID server}
The PID server is maintained by the PID provider, which identifies objects of a particular PID type. The PID types we cover in our proof of concept are highlighted in section \ref{pid-types}. For our proof of concept we have set up our own Handle PID server with Cordra software \cite{cor}. We got allotted the Handle prefix \texttt{20.5000.481} by the Handle registry to be used for object identification. In addition to this, we also used the resolver of the national library of the Netherlands for resolving URN's as well as PANGAEA, which resolves DOI's for our proof of concept.

\subsubsection{Client}\label{client}
The role of the client is to transparently make the user either retrieve the requested object from the corresponding PID provider or from the NDN network. The client receives the PID based on the user's input, which can be any kind of PID type. The client then opens a socket and sends the user request to the gateway, which corresponds with step 1 and 2 in figure \ref{fig:sdc_model}. After the gateway does the translation, it send back either a translated NDN name from the PID or a link to the PID server to the client. This depends whether the object is published in the NDN network or not. If it is, the client receives the NDN name. If not, it receives the link to the PID server for retrieving the object. This corresponds to step 6a in figure \ref{fig:sdc_model}.

If an NDN name is sent back to the client, the client then requests the object from the NDN network. This is shown as step 7a and step 8 till 10 in figure \ref{fig:sdc_model}. For object retrieval in NDN we used the tool \texttt{ndncatchunks} part of the \texttt{ndn-tools} software \cite{ndn-tools}, which makes use of the libraries of the NDN-CXX application \cite{ndn-tools}. 
If the object is not in NDN and a PID link is sent back, the client requests the object by its PID link by sending the request to the PID server for object retrieval. This corresponds to step 7c in figure \ref{fig:sdc_model}. Object retrieval is done at client side, otherwise the gateway has to retrieve the object first before sending it to the client. 
As in the case of retrieving the object from the PID server to publish it in NDN, the client should then wait before the gateway has retrieved and cached the object in NDN. 
In the case that the object is already published in NDN, the gateway server only concerns itself with translation and sending back the NDN name. This eliminates unnecessary load on the gateway.

Transparency for the user can be achieved. This is accomplished by combining the code for retrieving the object from the PID server with the code we have used to retrieve the object from NDN at the client side. Furthermore, a statement needs to be added at the gateway to check if the object is already published in NDN with the \texttt{ndnping} tool \cite{ndn-tools} for example as described in section \ref{gw}. 

This way, the user might not even be aware of an NDN network. The user only enters a PID as input for the client, without specifying the link of the web resolver that handles the PID. The user gets redirected automatically by the client to either the PID server or NDN network. No further user input is needed as everything is handled by the gateway, which will be discussed below in section \ref{gw}. 
Unlike previous designs, which require user input after the client has translated the PID to NDN name \cite{ndn-app-aware}. 

\subsubsection{Gateway}\label{gw}
The gateway used in our design follows the principles of Karakannas by not doing the PID to NDN translation on client side, as the client side has to be updated every time a new PID type scheme is introduced. For our proof of concept, we use a translation server called the PID to NDN gateway. The PID to NDN gateway implements the translation of different PID types and sends the translated name back to the client. Furthermore, we identify PID types based on pattern matching as described by Mousa \cite{ndn-app-aware}. We have also explored the N2T resolver, which resolves different PID types by stating the PID type that needs to be implemented along with the pattern of the PIDs' scheme \cite{n2t}.

%Translation
The gateway is responsible for translating a PID name to NDN name and checks if the requested object is already published in NDN. 
The first responsibility is translating the PID it receives from the client to an NDN name. This can be any PID type. In our proof of concept we have implemented the Handle PID type scheme of the Handle PID server we setup. In addition to this, we have also implemented the URN PID type scheme of the national library of the Netherlands, as well as the DOI type scheme of PANGAEA. The gateway receives a PID from the client, without the link of the corresponding web resolver of the PID type. Based on pattern matching of the PID type scheme, the gateway detects what kind of PID type it has to deal with and appends the corresponding link of the web resolver of the PID type it receives. 
%Pattern matching is done based on the patterns of 
The patterns of most standardized PID type schemes are maintained in the ePIC DTR and can be used for implementation in the gateway we propose \cite{dtr}. 

%Naming:
Furthermore, the next responsibility of the gateway is to check if the object is already published in NDN. This can be done with a simple ping to the object with \texttt{ndnping}, part of the \texttt{ndn-tools} software used in our proof of concept \cite{ndn-tools}. 
If the object is available in NDN, the gateway sends the translated NDN name back to the client where the client retrieves the object from NDN. This is shown in figure \ref{fig:seq_ndn}, where the Handle PID type is used as an example. If the object is not available in NDN, the gateway sends back the PID link to the client. The client then retrieves the object by its PID link from the PID server as shown in figure \ref{fig:seq_pid}. The PID link that is sent back to the client contains the PID and the link to the corresponding PID web resolver. This happens before the gateway retrieves the object to cache it in NDN, otherwise the client needs to wait for this as discussed in section \ref{client}. For our proof of concept we use the tool \texttt{ndnputchunks}, which is part of the \texttt{ndn-tools} software for caching objects in the NDN network \cite{ndn-tools}.

Transparency can be achieved, if the code we used in our setup to retrieve the object from NDN is combined with the code we used to retrieve the object from the PID server. This is done by including a statement to check if the object is already published in NDN with \texttt{ndnping} as we mentioned. 

To translate the matched PID type to an NDN name, one has to take into account the scheme of the PID type and the hierarchical way it has to be divided in. In our proof of concept, for deriving an NDN name from each discussed PID type, a prefix is added before the PID name. Such as \texttt{/ndn/handle} for Handle objects, \texttt{/ndn/doi} for DOI objects and \texttt{/ndn/urn} for URN objects. The delimiters of the PID types are replaced by a slash \texttt{("/")}. In PANGAEA, specific columns and parameters can be requested  to retrieve a particular part from an object. For example the following DOI object in PANGAEA \url{https://doi.pangaea.de/10.1594/PANGAEA.842227&columns=1,,2,3&filterParameterValue=Station,TARA_100} gets translated to the NDN name \url{/ndn/doi/10.1594/PANGAEA.842227/attrib+ndn+1,2,3+Station,TARA_100} \cite{ndn-app-aware}. Here it is shown that not only the delimiters are replaced to translate it into an NDN name, also The columns and parameters are translated to an NDN name. This is done by hierarchically dividing it in \texttt{/attrib+ndn} and appending the columns and parameters with a plus ("+") (where the requested columns and parameters are separated by a comma(",")). Furthermore, the web resolver link is also stripped after translation.

The web resolver link is not used for deriving the NDN name.
By excluding the web resolver link from the NDN name, duplications will not occur in NDN as the NDN name is only derived from the PID. A PID always translates to the same name in NDN this way. 
If the PID object is moved to another web resolver, only the link of the web resolver has to be changed in the gateway. 

Below, the translation of a Handle PID to an NDN name is shown, the Handle name is hierarchically divided into its PID type, authority and sub-authority.
\vspace{1em}
\begin{lstlisting}[frame=single,gobble=0,basicstyle=\scriptsize\ttfamily]
<user>@consumer-1:~/python-ndn$ python3 server_pid.py
Waiting for client
PID from connected user: 20.500.481/sub-auth/object1
PID type: Handle
NDN name from Handle: /ndn/handle/20.500.481/sub-auth/object1
\end{lstlisting}

The translation of a URN PID to an NDN name is shown below. The national library of the Netherlands chooses to assign a PID based on the year when the object has been published. 
This way, objects in NDN can be hierarchically divided in year, months or days for example.
\vspace{1em}
\begin{lstlisting}[frame=single,gobble=0,basicstyle=\scriptsize\ttfamily]
<user>@consumer-1:~/python-ndn$ python3 server_pid.py
Waiting for client
PID from connected user: anp:1938:10:01:2:mpeg21
PID type: URN
NDN name from URN: /ndn/urn/anp/1938/10/01/2/mpeg21
\end{lstlisting}

Metadata can be used if missing gaps need to be filled in for dividing the PID in the NDN hierarchy, as described by Olschanowsky et. al. \cite{ndn-clim} in section \ref{introduction-related-work}. 
Filling in missing gaps highly depends on which way the metadata is served by the providers of the different PID types as discussed in section \ref{pid-types}. Thus, if metadata is used, a parser has to be implemented in the gateway. This could be a XML, JSON or another kind of parser depending on the PID provider. In our proof of concept we have implemented a XML parser for URN's and a JSON parser for Handles.

There is no address exhaustion problem in NDN as the NDN namespace is unbounded \cite{ndn-nspace}. But worth mentioning is to keep in mind that using long NDN names degrades performance with many interests as described by Yuan et al. \cite{yuan2012scalable} 
which is mentioned in section \ref{introduction-related-work}.

\begin{figure}[H]
%\flushleft
    \centering
%    \makebox[0pt]
    \caption{Handle PID request\label{fig:seq_pid}}
\end{figure}

\begin{figure}[H]
\includegraphics[scale=0.75]{Images/ndn_req.png}
\caption{Handle NDN request}
\label{fig:seq_ndn}
\end{figure}

\subsection{Results}
The outcome of implementing our design in a proof of concept shows that our principles can be adhered.
%that PID interoperability of different PID types within the NDN namespace is possible. 
Making the translation transparent to the user is possible, as the gateway is responsible for PID to NDN translation and the object retrieval is taken care for by the client. This can be achieved by combining the code we used for retrieving the object from either the PID server or NDN. Translation is achieved by first recognizing the PID type based on pattern matching and then hierarchically divide the PID to an NDN name. Support for multiple PID types is also achieved by adding the scheme of the PID types at the gateway, which makes it also easily extensible to support future PID types. By adding PID types at the gateway, we overcome the hurdle of updating the client side with the scheme of newly introduced PID types each time when a new PID type is introduced.