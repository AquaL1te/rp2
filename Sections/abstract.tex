\begin{abstract}
Research clouds contain diverse and large datasets, this data is published by the use of \glspl{pid}. The current paradigm utilizes data transmissions by the means of host-to-host connections (\gls{ip}), where every data request from the consumer are answered with a data transfer from the source (the producer). This approach can potentially cause congestion and delays with many data consumers. \gls{ndn} is a data centric approach where unique data, once requested, is stored on intermediate hops in the network. Consecutive requests for that unique data object are then made available by these intermediate hops (caching). This approach distributes traffic load more efficient and reliable compared to host-to-host connection oriented techniques \cite{ndn-ccr}. Furthermore, one of the most important technical challenges is to incorporate interoperability between NDN and the different \gls{pid} naming schemas. These naming schemas are used by data providers within these data infrastructures for sharing and identifying digital objects. This research investigated how identification services and transmission services can be better integrated by the use of \gls{ndn}. In order to create this integration, a translation between the different naming schemas was developed and made extendable for future \gls{pid} types. Furthermore, a method was researched and applied for planning and scaling an \gls{ndn} with scalability in mind.

\end{abstract}

% The work you did with NDN/PID is to investigate how identification services and transmission services can be better integrated. The current world: \gls{pid} is publication, and transmission is based on other network protocols. Our motivtion is to investigate if PID, caching, digital object discovery, routine etc. can be all implemented using the \gls{ndn} technology





%SeaDataCloud is a distributed infrastructure to manage large and diverse sets of data about seas and oceans. This Pan-European network offers data access to support scientific workflows which varies from climate change prediction to offshore engineering \cite{sdc}.
%Different independent organizations push data into this infrastructure which are then automatically and manually curated to ensure correct data formats. A \gls{pid} is then assigned when it’s stored in the catalog. This \gls{pid} ensures that the data can be identified, independent of its location in the infrastructure \cite{icn-survey, icn-bd}. 
%CS3 is another case of a data infrastructure, which includes around hundred institutions and companies from around the world and this number keeps growing.
%Their most important technical challenge is incorporating introperability of the different naming schemas being used within CS3.