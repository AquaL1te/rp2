\section{Technical overview}\label{tech-oview}
Short intro about this section - its about giving the reader a detailed overview of the technology used, to showcase that we know what we talk about and to demonstrate that these technologies fit into answering the research question.

\subsection{PID types and naming scheme}\label{pid-types}
Over the years, different kinds of PIDs have emerged. The most widely used PID types are Handle (first major PID type introduced 
in 1994), Digital Object Identifier (DOI), Persistent URL (PURL), Uniform Resource Name (URN) and 
Archival Resource Key (ARK) \cite{pid-oview, odin, hdl}.

Every PID type makes use of the same hierarchical naming scheme, which starts with a PID type identifier,
 such as \texttt{"urn"}, \texttt{"handle"} or \texttt{"doi"}, followed by some kind of delimiter. The PID type identifier is usually embedded in the URL of the PID resolver naming authority 
such as \texttt{http://hdl.handle.net/} for resolving Handle PIDs, \texttt{https://doi.pangaea.de/} for resolving DOI PIDs or \texttt{http://resolver.kb.nl/resolve?urn=} for resolving URN PIDs, followed by the PID of the digital object. This means that PIDs are usually associated with a resolver via a URL \cite{ids, icn-bd}.

After this the naming authority is defined (which can be seen as a prefix of a digital object), followed again 
by some kind of delimiter. At last there is a Namespace Specific String which is the identity of a digital object within the naming authority, which syntax depends on the naming
authority.
%A more in-depth description of PIDs can be found in \cite{icn-bd} and \cite{pid-oview}.
For a more in-depth description of PIDs we refer to \cite{icn-bd} and \cite{pid-oview}.

Talk about how DOI is interoperable within Handle?

\subsubsection{ARK}
Bla bla

\subsubsection{DOI}
Bla bla

\subsection{NDN}
\label{overview-ndn}
\subsubsection{Caching}
Bla bla, strategies (also cache replacement strategies), which one is best?

\subsubsection{Forwarding}
Bla bla - Strategies, OSFPN/NLSR, which one is best?

\subsection{McCabe}
\label{overview-mccabe}
Describe his method in more detail, explain adaptions towards ndn, if applicable.

\subsection{TOSCA}
Why use it, why did we use it? Are there alternatives? Compare them shortly - if needed
Highlight popularity, Google, RedHat, Canonical, etc. - to emphasize that TOSCA won't go away soon, but keeps growing
Merge McCabe's method and tosca as one

\subsubsection{Orchestrators}
Describe DRIP, OpenStack and an orchestrator in general

\subsection{Virtualization}
\subsubsection{Docker}
What is it, why use it, why do we use it and how?

\subsubsection{Kubernetes}
What is it, why use it, why do we use it and how?